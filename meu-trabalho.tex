% -----------------------------------------------------------------------------
%       Centro Federal de Educação Tecnológica de Minas Gerais - CEFET-MG
%
%   Modelo de trabalho acadêmico em conformidade com as normas da ABNT
%   (Tese de Doutorado, Dissertação de Mestrado ou Projeto de Qualificação)
%
%    Projeto hospedado em: https://github.com/cfgnunes/latex-cefetmg
%
%    Autores: Cristiano Fraga Guimarães Nunes <cfgnunes@gmail.com>
%             Henrique Elias Borges <henrique@lsi.cefetmg.br>
%             Denise de Souza <densouza@gmail.com>
%             Lauro César <http://www.abntex.net.br/>
% -----------------------------------------------------------------------------

\documentclass[%
    %twoside,                   % Impressão em frente (anverso) e verso
    oneside,                    % Impressão apenas no anverso
]{cefetmg}

\usepackage[%
    alf,
    abnt-emphasize=bf,
    bibjustif,
    recuo=0cm,
    abnt-doi=expand,            % Expande um endereço iniciado com doi: para http://dx.doi.org/
    abnt-url-package=url,       % Utiliza o pacote url
    abnt-refinfo=yes,           % Utiliza o estilo bibliográfico abnt-refinfo
    abnt-etal-cite=3,
    abnt-etal-list=3,
    abnt-thesis-year=final
]{abntex2cite}                  % Configura as citações bibliográficas conforme a norma ABNT

% -----------------------------------------------------------------------------
% Pacotes utilizados
% -----------------------------------------------------------------------------
\usepackage[utf8]{inputenc}                                 % Codificação do documento
\usepackage[T1]{fontenc}                                    % Seleção de código de fonte
\usepackage{booktabs}                                       % Réguas horizontais em tabelas
\usepackage{color, colortbl}                                % Controle das cores
\usepackage{float}                                          % Necessário para tabelas/figuras em ambiente multicolunas
\usepackage{graphicx}                                       % Inclusão de gráficos e figuras
\usepackage{icomma}                                         % Uso de vírgulas em expressões matemáticas
\usepackage{indentfirst}                                    % Recua o primeiro parágrafo de cada seção
\usepackage{microtype}                                      % Melhora a justificação do documento
\usepackage{multirow, array}                                % Permite tabelas com múltiplas linhas e colunas
\usepackage{subeqnarray}                                    % Permite subenumeração de equações
\usepackage{verbatim}                                       % Permite apresentar texto tal como escrito no documento, ainda que sejam comandos Latex
\usepackage{amsfonts, amssymb, amsmath}                     % Fontes e símbolos matemáticos
\usepackage[algoruled, portuguese]{algorithm2e}             % Permite escrever algoritmos em português
\usepackage[scaled]{helvet}                                 % Usa a fonte Helvetica
%\usepackage{times}                                         % Usa a fonte Times
%\usepackage{palatino}                                      % Usa a fonte Palatino
%\usepackage{lmodern}                                       % Usa a fonte Latin Modern
%\usepackage[bottom]{footmisc}                              % Mantém as notas de rodapé sempre na mesma posição
%\usepackage{ae, aecompl}                                   % Fontes de alta qualidade
%\usepackage{latexsym}                                      % Símbolos matemáticos
%\usepackage{lscape}                                        % Permite páginas em modo "paisagem"
%\usepackage{picinpar}                                      % Dispor imagens em parágrafos
%\usepackage{scalefnt}                                      % Permite redimensionar tamanho da fonte
%\usepackage{subfig}                                        % Posicionamento de figuras
%\usepackage{upgreek}                                       % Fonte letras gregas

% Redefine a fonte para uma fonte similar a Arial (fonte Helvetica)
\renewcommand*\familydefault{\sfdefault}

% -----------------------------------------------------------------------------
% Configurações de aparência do PDF final
% -----------------------------------------------------------------------------
\makeatletter
\hypersetup{%
    portuguese,
    colorlinks=true,            % true: "links" coloridos; false: "links" em caixas de texto
    linkcolor=blue,             % Define cor dos "links" internos
    citecolor=blue,             % Define cor dos "links" para as referências bibliográficas
    filecolor=blue,             % Define cor dos "links" para arquivos
    urlcolor=blue,              % Define a cor dos "hiperlinks"
    breaklinks=true,
    pdftitle={\@title},
    pdfauthor={\@author},
    pdfkeywords={abnt, latex, abntex, abntex2}
}
\makeatother

% Altera o aspecto da cor azul
\definecolor{blue}{RGB}{41,5,195}

% Redefinição de labels
\renewcommand{\algorithmautorefname}{Algoritmo}
\def\equationautorefname~#1\null{Equa\c c\~ao~(#1)\null}

% Cria o índice remissivo
\makeindex

% Hifenização de palavras que não estão no dicionário
\hyphenation{%
    qua-dros-cha-ve
    Kat-sa-gge-los
}

% -----------------------------------------------------------------------------
% Inclui os arquivos do trabalho acadêmico
% -----------------------------------------------------------------------------

% Insere e constrói alguns elementos pré-textuais para gerar capa, folha de rosto e folha de aprovação
% -----------------------------------------------------------------------------
% Capa
% -----------------------------------------------------------------------------

% -----------------------------------------------------------------------------
% ATENÇÃO:
% Caso algum campo não se aplique ao seu documento - por exemplo, em seu trabalho
% não houve coorientador - não comente o campo, apenas deixe vazio, assim: \campo{}
% -----------------------------------------------------------------------------

% -----------------------------------------------------------------------------
% Dados do trabalho acadêmico
% -----------------------------------------------------------------------------

\titulo{Desenvolvimento de Aplicação Móvel Nativa Cross Platfrom com Flutter}
\autor{Rafael Cruz Teixeira}
\local{Belo Horizonte}
\data{Dezembro de 2017} % Normalmente se usa apenas mês e ano

% -----------------------------------------------------------------------------
% Natureza do trabalho acadêmico
% Use apenas uma das opções: Tese (p/ Doutorado), Dissertação (p/ Mestrado) ou
% Projeto de Qualificação (p/ Mestrado ou Doutorado), Trabalho de Conclusão de
% Curso (Graduação)
% -----------------------------------------------------------------------------

\projeto{Trabalho de Conclusão de Curso}

% -----------------------------------------------------------------------------
% Título acadêmico
% Use apenas uma das opções:
% - Se a natureza for Tese, coloque Doutor
% - Se a natureza for Dissertação, coloque Mestre
% - Se a natureza for Projeto de Qualificação, coloque Mestre ou Doutor conforme o caso
% - Se a natureza for Trabalho de Conclusão de Curso, coloque Bacharel
% -----------------------------------------------------------------------------

\tituloAcademico{}

% -----------------------------------------------------------------------------
% Área de concentração e linha de pesquisa
% Observação: Indique o nome da área de concentração e da linha de pesquisa do Programa de Pós-graduação
% nas quais este trabalho se insere
% Se a natureza for Trabalho de Conclusão de Curso, deixe ambos os campos vazios
% -----------------------------------------------------------------------------

\areaconcentracao{}
\linhapesquisa{}

% -----------------------------------------------------------------------------
% Dados da instituição
% Observação: A logomarca da instituição deve ser colocada na mesma pasta que foi colocada o documento
% principal com o nome de "logoInstituicao". O formato pode ser: pdf, jpf, eps
% Se a natureza for Trabalho de Conclusão de Curso, coloque em "programa' o nome do curso de graduação
% -----------------------------------------------------------------------------

\instituicao{Pontifícia Universidade Católica de Minas Gerais}
\programa{Programa de Pós-graduação em Arquitetura de Sistemas Distribuídos}
\logoinstituicao{0.4}{./04-figuras/logo-puc.jpg} % \logoinstituicao{<escala>}{<nome do arquivo>}

% -----------------------------------------------------------------------------
% Dados do(s) orientador(es)
% -----------------------------------------------------------------------------

\orientador{Tadeu dos Reis Faria}
%\orientador[Orientadora:]{Nome da orientadora}
\instOrientador{PUC Minas}

\coorientador{}
%\coorientador[Coorienta dora:]{Nome da coorientadora}
\instCoorientador{}

\include{./01-elementos-pre-textuais/folha-rosto}
\include{./01-elementos-pre-textuais/folha-aprovacao}

\begin{document}

% Insere os elementos pré-textuais
\pretextual
\imprimircapa                                               % Comando para imprimir Capa

% -----------------------------------------------------------------------------
% Resumo
% -----------------------------------------------------------------------------

\begin{resumo}
    

    %\textbf{Palavras-chave}: Modelo Latex. Trabalho acadêmico monográfico. Normas ABNT. Outra palavra.
\end{resumo}

% -----------------------------------------------------------------------------
% Escolha de 3 a 6 palavras ou termos que descrevam bem o seu trabalho. As palavras-chaves são utilizadas para indexação.
% A letra inicial de cada palavra deve estar em maiúsculas. As palavras-chave são separadas por ponto.
% -----------------------------------------------------------------------------
             % Resumo na língua vernácula
\include{./01-elementos-pre-textuais/sumario}               % Sumário

% Insere os elementos textuais
\textual
% -----------------------------------------------------------------------------
% Introdução
% -----------------------------------------------------------------------------
\chapter{Referencial Teórico}

\section{Padrões de Desenvolvimento Mobile}
\label{sec:whatisflutter}
Existem maneiras diferentes de desenvolver aplicativos para os dois principais SO’s (sistemas operacionais) do mercado de dispositivos móveis, Android e iOS. Dentre as principais, podemos citar:


\begin{itemize}
    \item \textbf{Desenvolvimento Nativo} 
    \item \textbf{WebApps e PWA’s (progressive web apps)} 
    \item \textbf{Desenvolvimento Cross-Platform Hybrid} 
    \item \textbf{Desenvolvimento Cross-Platform Native} 
\end{itemize}

O desenvolvimento nativo consiste em desenvolver a aplicação diretamente para a plataforma em que ela vai rodar.
Caso seja necessário que a aplicação rode em mais de uma plataforma, uma aplicação precisará ser escrita para cada 
sistema operacional.

Esse tipo de desenvolvimento proporciona melhor desempenho, permite controle completo sobre a 
aplicação e acesso direto a todos os recursos que o sistema operacional provê. 
Como os SO’s são fundamentalmente diferentes, é necessário escrever um código 
específico para que a aplicação rode em cada um deles. \citeonline{hibridoNativo1} \citeonline{ionic1} \citeonline{barreto}

O desenvolvimento das WebApps e aplicações Híbridas surgiu como uma 
alternativa ao desenvolvimento nativo devido à necessidade de criar 
aplicações de forma mais rápida, sem a necessidade de se preocupar 
com as especificidades de cada plataforma em que o programa vai rodar
\citeonline{ionic1}. 
Um WebApp é desenvolvido utilizando HTML e Javascript, e roda no browser do dispositivo móvel, 
por isso torna-se independente da plataforma. Porém existem diversas limitações relativas 
ao que uma aplicação desse tipo pode fazer. O acesso aos recursos de sistema operacional 
e hardware são limitados ao que os SO’s permitem que o navegador web faça. 
\citeonline{salesforce} \citeonline{barreto}

Com a evolução dos browsers, foi possível dar mais poder às WebApps, 
e consequentemente melhorar a experiência dos usuários. 
Os PWA’s representam essa evolução. A aplicação PWA continua sendo 
executada no contexto de um browser, porém pode ser instalada no dispositivo 
do usuário, possui mais acesso a recursos do SO como notificações e armazenamento 
offline de dados, por exemplo. \citeonline{madeInWeb} \citeonline{pwa1}

Aplicações Híbridas seguem a mesma ideia das WebApps e PWA’s: desenvolver 
apenas um código e conseguir rodar a aplicação nas principais plataformas disponíveis.
Porém, mesmo carregando a mesma ideia, são abordagens diferentes [MENDES et al. 2014].  
As WebApps e PWA’s são feitas para rodar no navegador. 
Já as aplicações híbridas são feitas para serem instaladas nativamente no sistema operacional
 de cada dispositivo e rodam utilizando WebViews. 
 \citeonline{madeInWeb} \citeonline{pwa1} \citeonline{barreto}
O desenvolvimento híbrido consiste em escrever as aplicações em HTML, CSS e JavaScript, 
e um framework gera uma aplicação nativa que roda em uma WebView no dispositivo, 
e fornece a comunicação entre o Javascript e HTML codificados com os recursos do sistema operacional.
 Essa prática permite flexibilidade no desenvolvimento, maior proximidade com o nativo, 
 devido à possibilidade de acesso a mais recursos dos sistemas operacionais, 
 um desempenho piorado, por necessitar de um intermediário para transmitir as 
 instruções da View para os Sistemas Operacionais e o aplicativo fica restrito 
 às capacidades do Framework de desenvolvimento híbrido escolhido \citeonline{salesforce} \citeonline{barreto}.  
 Existem opções de mercado como PhoneGap, Apache Cordova, e Ionic. \citeonline{xamarin} \citeonline{cordova} \citeonline{ionic2}

Como quarta opção de forma de desenvolvimento, temos a construção de aplicações
 Cross Platform nativas utilizando SDKs como o React Native e o Flutter, 
 que geram os executáveis para binários nativos em cada plataforma. 
 \citeonline{monteiro} \citeonline{barreto}. 
Essa forma de desenvolvimento permite uma grande integração com o 
Sistema Operacional e acesso direto aos recursos de hardware do dispositivo móvel em questão.

O objetivo deste trabalho é estudar e viabilizar o desenvolvimento de 
uma aplicação móvel híbrida nativa utilizando o SDK Flutter, desenvolvido pelo Google.

\section{O que é Flutter?}
\label{sec:whatisflutter}
Flutter é um SDK de código aberto criado pelo Google, em 2017, 
que permite desenvolvimento de aplicações nativas para iOS e Android 
utilizando código escrito em apenas uma linguagem: Dart. 
O Flutter é capaz de gerar código executável verdadeiramente nativo para Android e iOS, 
sem a utilização de intermediários para transmitir instruçòes ao SO, 
obtendo com isso desempenho superior ao de aplicações Cross Platform, sejam Híbridas ou Nativas. 
\citeonline{flutter} \citeonline{barreto}

\section{O que é DART?}
\label{sec:whatisdart}
Dart é uma linguagem de programação criada pelo Google em 2011, 
otimizada para desenvolvimento de aplicações web e mais 
recentemente também para aplicações móveis. É uma linguagem compilada, 
fortemente tipificada e orientada a objetos. \citeonline{dart} \citeonline{abranches}

                % Introdução

% Insere os elementos pós-textuais
\postextual
\include{./03-elementos-pos-textuais/referencias}           % Referências

\end{document}
