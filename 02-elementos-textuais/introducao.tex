% -----------------------------------------------------------------------------
% Introdução
% -----------------------------------------------------------------------------
\chapter{Referencial Teórico}

\section{Padrões de Desenvolvimento Mobile}
\label{sec:whatisflutter}
Existem maneiras diferentes de desenvolver aplicativos para os dois principais SO’s (sistemas operacionais) do mercado de dispositivos móveis, Android e iOS. Dentre as principais, podemos citar:


\begin{itemize}
    \item \textbf{Desenvolvimento Nativo} 
    \item \textbf{WebApps e PWA’s (progressive web apps)} 
    \item \textbf{Desenvolvimento Cross-Platform Hybrid} 
    \item \textbf{Desenvolvimento Cross-Platform Native} 
\end{itemize}

O desenvolvimento nativo consiste em desenvolver a aplicação diretamente para a plataforma em que ela vai rodar.
Caso seja necessário que a aplicação rode em mais de uma plataforma, uma aplicação precisará ser escrita para cada 
sistema operacional.

Esse tipo de desenvolvimento proporciona melhor desempenho, permite controle completo sobre a 
aplicação e acesso direto a todos os recursos que o sistema operacional provê. 
Como os SO’s são fundamentalmente diferentes, é necessário escrever um código 
específico para que a aplicação rode em cada um deles. \citeonline{hibridoNativo1} \citeonline{ionic1} \citeonline{barreto}

O desenvolvimento das WebApps e aplicações Híbridas surgiu como uma 
alternativa ao desenvolvimento nativo devido à necessidade de criar 
aplicações de forma mais rápida, sem a necessidade de se preocupar 
com as especificidades de cada plataforma em que o programa vai rodar
\citeonline{ionic1}. 
Um WebApp é desenvolvido utilizando HTML e Javascript, e roda no browser do dispositivo móvel, 
por isso torna-se independente da plataforma. Porém existem diversas limitações relativas 
ao que uma aplicação desse tipo pode fazer. O acesso aos recursos de sistema operacional 
e hardware são limitados ao que os SO’s permitem que o navegador web faça. 
\citeonline{salesforce} \citeonline{barreto}

Com a evolução dos browsers, foi possível dar mais poder às WebApps, 
e consequentemente melhorar a experiência dos usuários. 
Os PWA’s representam essa evolução. A aplicação PWA continua sendo 
executada no contexto de um browser, porém pode ser instalada no dispositivo 
do usuário, possui mais acesso a recursos do SO como notificações e armazenamento 
offline de dados, por exemplo. \citeonline{madeInWeb} \citeonline{pwa1}

Aplicações Híbridas seguem a mesma ideia das WebApps e PWA’s: desenvolver 
apenas um código e conseguir rodar a aplicação nas principais plataformas disponíveis.
Porém, mesmo carregando a mesma ideia, são abordagens diferentes [MENDES et al. 2014].  
As WebApps e PWA’s são feitas para rodar no navegador. 
Já as aplicações híbridas são feitas para serem instaladas nativamente no sistema operacional
 de cada dispositivo e rodam utilizando WebViews. 
 \citeonline{madeInWeb} \citeonline{pwa1} \citeonline{barreto}
O desenvolvimento híbrido consiste em escrever as aplicações em HTML, CSS e JavaScript, 
e um framework gera uma aplicação nativa que roda em uma WebView no dispositivo, 
e fornece a comunicação entre o Javascript e HTML codificados com os recursos do sistema operacional.
 Essa prática permite flexibilidade no desenvolvimento, maior proximidade com o nativo, 
 devido à possibilidade de acesso a mais recursos dos sistemas operacionais, 
 um desempenho piorado, por necessitar de um intermediário para transmitir as 
 instruções da View para os Sistemas Operacionais e o aplicativo fica restrito 
 às capacidades do Framework de desenvolvimento híbrido escolhido \citeonline{salesforce} \citeonline{barreto}.  
 Existem opções de mercado como PhoneGap, Apache Cordova, e Ionic. \citeonline{xamarin} \citeonline{cordova} \citeonline{ionic2}

Como quarta opção de forma de desenvolvimento, temos a construção de aplicações
 Cross Platform nativas utilizando SDKs como o React Native e o Flutter, 
 que geram os executáveis para binários nativos em cada plataforma. 
 \citeonline{monteiro} \citeonline{barreto}. 
Essa forma de desenvolvimento permite uma grande integração com o 
Sistema Operacional e acesso direto aos recursos de hardware do dispositivo móvel em questão.

O objetivo deste trabalho é estudar e viabilizar o desenvolvimento de 
uma aplicação móvel híbrida nativa utilizando o SDK Flutter, desenvolvido pelo Google.

\section{O que é Flutter?}
\label{sec:whatisflutter}
Flutter é um SDK de código aberto criado pelo Google, em 2017, 
que permite desenvolvimento de aplicações nativas para iOS e Android 
utilizando código escrito em apenas uma linguagem: Dart. 
O Flutter é capaz de gerar código executável verdadeiramente nativo para Android e iOS, 
sem a utilização de intermediários para transmitir instruçòes ao SO, 
obtendo com isso desempenho superior ao de aplicações Cross Platform, sejam Híbridas ou Nativas. 
\citeonline{flutter} \citeonline{barreto}

\section{O que é DART?}
\label{sec:whatisdart}
Dart é uma linguagem de programação criada pelo Google em 2011, 
otimizada para desenvolvimento de aplicações web e mais 
recentemente também para aplicações móveis. É uma linguagem compilada, 
fortemente tipificada e orientada a objetos. \citeonline{dart} \citeonline{abranches}

